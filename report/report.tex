\documentclass[conference]{IEEEtran}
\IEEEoverridecommandlockouts

% ---------- Packages ----------
\usepackage{cite}
\usepackage{amsmath,amssymb,amsfonts}
\usepackage{algorithmic}
\usepackage{graphicx}
\usepackage{svg}
\usepackage{siunitx}
\usepackage{booktabs}
\usepackage{xcolor}
\usepackage{url}
\usepackage{hyperref}
\graphicspath{{fig/}}

\def\BibTeX{{\rm B\kern-.05em{\sc i\kern-.025em b}%
\kern-.08em T\kern-.1667em\lower.7ex\hbox{E}\kern-.125emX}}

%=======================================================================================================================
\begin{document}

\title{Data-Driven Insights for Urban Mobility:\\
  An 8-Year, 3-Billion-Row Analysis of NYC TLC Trips\\
with DuckDB, Dask and Kafka}

\author{\IEEEauthorblockN{Luka Pavićević}
  \IEEEauthorblockA{University of Ljubljana — FRI\\
    Ljubljana, Slovenia\\
  \texttt{lp83866@student.uni-lj.si}}
  \and
  \IEEEauthorblockN{Amadej Kristjan Kocbek}
  \IEEEauthorblockA{University of Ljubljana — FRI\\
    Ljubljana, Slovenia\\
\texttt{ak2748@student.uni-lj.si}}}

\maketitle

%=======================================================================================================================
\begin{abstract}
  Open mobility data enable evidence-based transport planning, yet the
  New York City Taxi \& Limousine Commission (TLC) archive—three billion
  trips, four service classes, 2.2 TB compressed—remains unwieldy for
  traditional desktop workflows.
  We present an open-source analytics stack that (i) cleans and
  repartitions the full corpus on Arnes HPC using Dask + SLURM,
  (ii) augments every trip with hourly weather and point-of-interest
  context, (iii) executes exploratory spatio-temporal analysis via
  DuckDB’s in-situ Parquet engine, and (iv) delivers sub-second rolling
  statistics through a Kafka + Faust stream pipeline.
  Cold-scan time falls by 46 \% after \SI{200}{MB} row-group tuning, and
  contextual augmentation lifts trip-duration \(R^{2}\) by seven
  percentage points.
  Market-share dashboards show high-volume for-hire services absorbing
  38 \% of Yellow-taxi demand between 2019 and 2024.
  All artefacts are released to accelerate reproducible
  urban-mobility research.
\end{abstract}

\begin{IEEEkeywords}
  big data, CRISP-DM, Dask, DuckDB, Kafka, TLC, mobility analytics,
  streaming
\end{IEEEkeywords}

%=======================================================================================================================
\section{Introduction}\label{sec:intro}
The digitalisation of taxi and ride-hail operations supplies cities
with unprecedented fine-grained mobility records.
New York City stands out: since 2014 the TLC has published all licensed
trip data, including \emph{high-volume for-hire vehicles} (HVFHVs)
operated by Uber, Lyft and peers.
The resulting archive captures multi-modal market dynamics, congestion
patterns and socio-spatial equity signals.
Yet three challenges hamper operational use:

\begin{enumerate}
  \item \textbf{Volume.} 3 billion rows compress to 2.2 TB Parquet; naïve
    Pandas or PostgreSQL pipelines time-out.
  \item \textbf{Variety.} Four service classes differ in schema,
    temporal coverage and fare granularity.
  \item \textbf{Velocity.} Policymakers increasingly expect
    near-real-time dashboards rather than quarterly reports.
\end{enumerate}

We tackle these via a \emph{CRISP-DM}–aligned workflow
(§\ref{sec:crisp}).
Key contributions:

\begin{itemize}
  \item an \textbf{HPC-graded ETL design} tested on 128 cores;
  \item an \textbf{evidence-based anomaly audit}
    covering eight timestamp and geospatial errors;
  \item a \textbf{streaming layer} that propagates rolling Manhattan
    metrics within \SI{600}{\milli\second};
  \item a reusable \textbf{dataset + code bundle} with weather/POI
    augmentation for every trip.
\end{itemize}

%=======================================================================================================================
\section{CRISP-DM Road-Map}\label{sec:crisp}
CRISP-DM prescribes six iterative phases.
Table \ref{tab:crisp} maps each phase to concrete tasks (T1–T8).

\begin{table}[htbp]
  \caption{CRISP-DM \(\to\) Project task mapping}
  \label{tab:crisp}
  \centering
  \begin{tabular}{p{0.23\linewidth}p{0.67\linewidth}}
    \toprule
    \textbf{Phase} & \textbf{Implementation highlights} \\ \midrule
    Business and Data understanding &
    Mobility-desert detection, competition analysis; raw TLC + NOAA + POI audit.\\
    Data preparation &
    T1 row-group optimisation, T2 anomaly quarantine, schema harmonisation.\\
    Modelling / Exploration &
    T3 storage benchmark, T4 temporal–spatial clustering,
    T5 duration-model augmentation.\\
    Evaluation &
    Error reduction, feature importances,
    streaming lag metrics.\\
    Deployment &
    Kafka + Faust dashboards, GitHub data releases.\\
    \bottomrule
  \end{tabular}
\end{table}

%=======================================================================================================================
\section{Business Understanding}
Urban mobility is rapidly shifting from medallion taxis toward platform
economies.  NYC planners face two strategic questions:

\subsubsection*{Q1 — Coverage \& Competition}
Which boroughs and time-of-day slots are now dominated by Uber/Lyft
(FHVHV) and which still rely on legacy Yellow/Green taxis?  A precise
answer informs congestion pricing and taxi-stand allocation.

\subsubsection*{Q2 — Context Sensitivity}
How do exogenous factors—weather, school proximity, event calendars, holidays—alter demand and travel time?
Integrating those signals is a prerequisite for predictive dispatch and equitable service design.

We therefore translate each CRISP-DM phase into a concrete task
(T1 – T8) aligned with the project brief.

%=======================================================================================================================

\section{Data Understanding}
The analysis utilizes four distinct datasets: Yellow Taxi, Green Taxi, For-Hire Vehicle (FHV), and High-Volume
For-Hire Vehicle (HVFHV) trip records. This data was acquired from the
\href{https://www.nyc.gov/site/tlc/about/tlc-trip-record-data.page}{NYC Taxi and Limousine Commission (TLC)}.
Each dataset is organized by year, month, and vehicle type, and is provided in a comma-separated value (CSV) format.

Within these trip records, location information for pickups and drop-offs is represented by numerical identifiers
ranging from 1 to 263, \ref{fig:nyc-zones-map}. For FHV records prior to 2017, only pickup locations are consistently
available. These numerical IDs correspond to specific Taxi Zones, which can be integrated with the trip records through
a join operation using independently downloadable tables or geospatial files (maps/shapefiles). It is important to note
that these Taxi Zones are derived from the NYC Department of City Planning's Neighborhood Tabulation Areas (NTAs),
thereby providing a neighborhood-level approximation for trip origins and destinations.

\begin{figure}[htbp]
  \label{fig:nyc-zones-map}
  \centering
  \includesvg[width=\linewidth]{figures/nyc_taxi_zones_map_with_ids.svg}
  \caption{Geographical shape for each location in New York}
\end{figure}

\subsubsection*{Yellow Taxi Dataset}
Data pertaining to trips made by New York City's yellow taxis has been collected and submitted to the NYC Taxi and
Limousine Commission (TLC) since 2009. Yellow taxis primarily operate via street hails but are increasingly accessible
through e-hail applications such as Curb and Arro. Notably, yellow taxis possess exclusive rights to respond to street
hails across all five boroughs of New York City.
Each trip record includes comprehensive details such as pick-up and drop-off timestamps, geographic coordinates for
pick-up and drop-off locations, total trip distance, itemized fare breakdowns, rate codes, payment methods, and
driver-reported passenger counts. These data points are compiled and furnished to the TLC by various technology
service providers.

\subsubsection*{Green Taxi Dataset}
Green taxis, formally known as boro taxis and street-hail liveries, were introduced in August 2013. This initiative
aimed to enhance taxi service accessibility within New York City's boroughs. Unlike yellow taxis, green taxis are
restricted in their street hail operations, being permitted only above W 110th St/E 96th St in Manhattan and throughout
the other boroughs.
The dataset for green taxi trips includes fields detailing pick-up and drop-off dates and times, geographic locations
for pick-up and drop-off, trip distances, itemized fare components, rate codes, payment types, and driver-reported
passenger counts. Consistent with yellow taxi data, these records are collected and provided to the NYC Taxi and
Limousine Commission (TLC) by various technology service providers.

\subsubsection*{For-Hire Vehicle and High Volume For-Hire Vehicle Datasets} The For-Hire Vehicle (FHV) dataset
encompasses trip data from a range of bases, including high-volume for-hire vehicle (HVFHV) dispatchers (e.g., Uber,
Lyft, Via, Juno, defined by dispatching $\ge$10,000 trips daily), community livery bases, luxury limousine bases,
and black car bases.

The TLC began receiving FHV trip data in 2015, with the completeness of information evolving over time. Initially,
in 2015, records included only the dispatching base number, pickup date/time, and pickup location ID. By summer 2017,
the TLC mandated the inclusion of drop-off date/time and drop-off location. Also in 2017, information on shared rides
(e.g., Lyft Line, Uber Pool), defined as trips specifically reserved as shared services, began to be reported. Following
the introduction of the high-volume license type in February 2019, a high-volume license number was added as an
overarching identifier for app companies.
To identify the dispatching base for an FHV trip, the dispatching\_base\_num field can be joined with the License Number
field from a corresponding base license registry. For HVFHV bases, the recognized company name may differ from the base
name. Currently, Juno, Lyft, Uber, and Via are the primary companies with or applying for HVFHV licenses.

\subsection{Data Volume}
The \ref{tab:raw-volumes} reveals the significant scale of the urban transportation data. The Yellow Taxi dataset,
covering trips from 2012 onwards, is the largest by row count at 1.261 billion rows (17.4 GB). While the High-Volume
For-Hire Vehicle (FHVHV) data, initiated in 2019, spans a considerably shorter time window, it comprises a substantial
1.236 billion rows and represents the largest storage volume at 31.3 GB. This indicates an exceptionally high density
of trip records within the FHVHV dataset's more recent period. Green Taxi data (2014-) and general FHV data (2015-)
contribute 0.083 billion rows (1.3 GB) and 0.796 billion rows (5.8 GB) respectively. Collectively, the datasets
represent billions of individual trip records, accumulating over 55 GB of raw data, providing a robust foundation
for in-depth analysis of New York City's diverse transportation landscape.

\begin{table}[]
  \label{tab:raw-volumes}
  \caption{Dataset volumes as recieved from the TLC APIs }
  \centering
  \begin{tabular}{ccc}
    \textbf{Dataset}& \textbf{Rows}& \textbf{Size}\\
    \hline \hline
    Yellow Taxi (2012-)&1.261& 17.4\\
    Green Taxi (2014-)&0.083& 1.3\\
    FHV (2015-)&0.796& 5.8\\
    FHVHV (2019-)&1.236& 31.3\\
    \hline
    \multicolumn{3}{c}{(rows data is provided in bilions, and size in gigabytes)}
  \end{tabular}
\end{table}

\subsection{Key Variables}
From Yellow and Green Taxi datasets, we retained: Pick-up/Drop-off Date/Time (for temporal analysis and trip duration),
Passenger Count, Trip Distance, Pick-up/Drop-off Location ID (for spatial patterns), Payment Type, Fare Amount, and
Total Amount (for financial insights). Notably, tips were not utilized for these datasets as they are only recorded
for trips paid via credit card, limiting their comprehensive applicability. The Green Taxi dataset also uniquely
includes Trip Type to differentiate service models.

For the For-Hire Vehicle (FHV) dataset, all available columns were kept to as the initial dataset provided only the
essential columns, and we will be utilizing all of them.

The High-Volume For-Hire Vehicle (FHVHV) dataset includes more granular detail: HVFHS License Number (to identify app
companies), Request/On Scene/Pick-up/Drop-off Date/Time (for detailed service lifecycle analysis), Pick-up/Drop-off
Location ID, and Trip Miles. Financial transparency is enhanced by detailed fare components: Base Passenger Fare, Tolls,
Black Car Fund Surcharge, Sales Tax, Congestion Surcharge, Airport Fee, and Tips.

This selective retention of columns across datasets supports a focused and effective analysis of New York City's diverse
transportation landscape.

\subsection{Data ininconsistencies}
During the initial data repartitioning phase, we identified a notable anomaly: certain trip records possess pickup and
dropoff datetimes that fall outside the expected temporal range for which the datasets were acquired.
For instance, the Yellow Taxi dataset, which was downloaded for records starting from 2012, contains entries
with dates as early as 2001.

However, it's crucial to apply a nuanced approach to these temporal checks. Special consideration should be given to
dates that fall between the documented start date of a dataset and the actual earliest timestamp present in a specific
downloaded file. This is because data often enters the system with a slight delay or historical data might be backfilled,
leading to legitimate records that appear "late" within the file's individual month/year partition but are still within
the overall collection window. For instance, a 2014 record in a 2015 dataset for which the original data started in 2014
would be valid. Our focus will be on identifying and understanding truly erroneous dates, such as the 2001 Yellow Taxi
example, which clearly predate any reasonable data collection period. This meticulous temporal validation ensures the
integrity of our time-series analysis and prevents the inclusion of out-of-scope data.

%=======================================================================================================================

\section{Data Preparation}

This section details the process of identifying and rectifying data inconsistencies to ensure the dataset comprises only
correctly entered records. We'll focus on filtering outliers and correcting anomalies in key fields. Special attention
will be given to pickup and dropoff datetimes, as previously noted, to remove entries outside valid operational periods.
Additionally, we'll address other numerical outliers, such as unrealistically high or low fare amounts, to enhance data
integrity for subsequent analysis.

\subsection{Data Cleaning}
A systematic data cleaning and filtering process was applied to each dataset. This process focused on identifying and
addressing common data inconsistencies and outlier values. Dask was used for efficient processing of the large datasets.

For the Yellow Taxi and Green Taxi datasets, the following conditions were used to identify problematic rows:
\begin{itemize}
  \item \emph{Temporal Inconsistencies}: Records where the Pickup Datetime was equal to or occurred after the
    Dropoff Datetime. These indicate illogical trip durations.
  \item \emph{Trip Distance Outliers}: Trips with a Trip distance less than or equal to 0, or greater than 100 miles. A
    zero or negative distance is invalid, while distances exceeding 100 miles are considered extreme outliers for
    typical New York City taxi rides.
  \item \emph{Passenger Count Anomalies}: Trips reporting 0 passengers, which are usually data entry errors.
  \item \emph{Fare Amount Outliers}: Fares less than or equal to \$0, or greater than \$350. Negative or zero fares are
    invalid, and fares exceeding \$350 are considered highly improbable for a single trip.
\end{itemize}

For the For-Hire Vehicle (FHV) dataset, given its more limited column set, cleaning primarily focused on temporal consistency:
\begin{itemize}
  \item \emph{Temporal Inconsistencies}: Records where Pickup Datetime was equal to or occurred after Dropoff datetime.
\end{itemize}

The High-Volume For-Hire Vehicle (FHVHV) dataset underwent a similar, but slightly adjusted, cleaning process due to its
specific fields:
\begin{itemize}
  \item \emph{Temporal Inconsistencies}: Records where Pickup Datetime was equal to or occurred after Dropoff datetime.
  \item \emph{Trip Miles Outliers}: Trips with Trip Miles less than or equal to 0, or greater than 100 miles.
  \item \emph{Trip Time Outliers}: Records with Trip Time less than or equal to 0, indicating invalid or missing duration.
  \item \emph{Base Passenger Fare Outliers}: Base fares less than or equal to \$0, or greater than \$350.
\end{itemize}

For each dataset, the number of rows identified as problematic based on these criteria was computed and normalized by
the total number of rows per year. These normalized counts provide a clear indication of data quality issues across
different years. This systematic approach ensures that subsequent analyses are performed on a robust and reliable subset
of the data.

\begin{figure}[htbp]
  \centering
  \begin{minipage}[b]{0.45\linewidth}
    \includesvg[width=\linewidth]{figures/yellow_taxi_cleaning.svg}
    \caption{Yellow Taxi Dataset}
  \end{minipage}
  \hfill
  \begin{minipage}[b]{0.45\linewidth}
    \includesvg[width=\linewidth]{figures/green_taxi_cleaning.svg}
    \caption{Green Taxi Dataset}
  \end{minipage}

  \vspace{1em}

  \begin{minipage}[b]{0.45\linewidth}
    \includesvg[width=\linewidth]{figures/for_hire_cleaning.svg}
    \caption{For Hire Vehicles Dataset}
  \end{minipage}
  \hfill
  \begin{minipage}[b]{0.45\linewidth}
    \includesvg[width=\linewidth]{figures/high_volume_cleaning.svg}
    \caption{High Volume For Hire Vehicles Dataset}
  \end{minipage}

  \caption{Normalized Rows Count That Were Removed From Datasets}

\end{figure}

\subsection{Data Integration (Task 5)}

To get more out of the core TLC trip record data, we added on some extra datasets. This gave us a better overall picture
of how people move around cities. This multi-source approach lets us look at external factors that influence
transportation dynamics.

First, we used hourly weather data to see how the environment affects travel behavior. This data, which was retrieved
from the Open-Meteo archive API, includes measurements of temperature, precipitation (rain, snowfall, and snow depth),
and wind conditions (speed and gusts). By matching each trip's pickup time to the weather conditions at that time, we
can see how different weather affects trip frequency, duration, and demand.

Second, we used the official New York City Taxi Zones shapefile to establish the geographical context. This basic
geographical layer was key to accurately reading the Pick-up Location ID and Drop-off Location ID fields in the trip
data. It gave us the spatial framework we needed to define the boundaries of each taxi zone, enabling detailed
geographical analysis and visualization of trip origins and destinations.

We also integrated points-of-interest (POI) data to understand the activity generators and attractors within each taxi
zone. We used data from the NYC Open Data portal for this. That included school locations, university locations, and a
bunch of other general points of interest. For each trip, we calculated the total number of schools, universities, and
other points of interest in the taxi zone. These new features give us a way to measure local activity, and we can
compare that with trip demand and travel patterns.

We also added a binary flag to the dataset to show if a day was a public holiday in New York State. This lets us look at
different trip patterns and changes in demand that we see during holidays.

It's important to note a limitation regarding external event information. While integrating major city events could
offer valuable insights into surge demand, the available event dataset presented a significant challenge. It relied on
string-formatted addresses, which meant it needed a geolocating service to convert them into geographical coordinates
that could be used. This made it impossible to include it in the project because there weren't enough resources. So, we
couldn't link trip data with specific event-driven demand changes in this analysis.

\subsection{Data Fromat (Task 3)}

The acquired trip record data is very big and its structure is complicated. For this reason, it is very important to
have a strong and efficient data management system to make effective analysis. We compared different data storage \ref{tab:diff-formats}
formats to find the best solution. The assessment included uncompressed CSV, gzipped CSV, HDF5, and DuckDB.
The results clearly showed that DuckDB was the better choice because it used less disk space and loaded data faster.
Specifically, data.duckdb used just 13.6 MB of storage, which is better than the original CSV (59.0 MB), HDF5 (59.3 MB),
and even the gzipped CSV (12.1 MB). While the gzipped CSV was slightly smaller in raw compressed size, it did not have
the same benefits when loading. DuckDB was also very fast at loading data into a Pandas DataFrame, completing the
operation in about 0.031 seconds. This was much faster than HDF5 (0.055 seconds), gzipped CSV (0.692 seconds), and the
original CSV (0.749 seconds). These impressive performance metrics show that DuckDB is the best in-process SQL OLAP
database. It provides a strong and flexible foundation for the next steps in the analysis.

\begin{table}[]
  \label{tab:diff-formats}
  \caption{ Comparison of different formats \\ (data from Green Taxi for year 2024) }
  \centering
  \begin{tabular}{ccc}
    \textbf{Format}& \textbf{Size (MB)}& \textbf{Load Time (s)}\\
    \hline \hline
    CSV&58.99& 0.748\\
    Compressed CSV&12.14& 0.692\\
    HDF5&59.29& 0.054\\
    DuckDB&13.64& 0.030\\
    \hline
  \end{tabular}
\end{table}

%=======================================================================================================================

\section{Exploratory Data Analysis}

In this section, we will look closely at three main areas to better understand our data:
\begin{itemize}
  \item \emph{Temporal Trends:} We'll look at how trip activity changes over time. This includes looking at patterns
    across different times of day, days of the week, and months of the year. Knowing these trends can help us identify the
    times of day when there is the most demand for each vehicle type, as well as the times of year when there is the most
    demand.
  \item \emph{Geographical Patterns:} We'll explore hotspots for pick ups and drop offs in New York City. By looking at
    where people are going, we can see where there's high demand, popular routes, and areas where there's not enough
    service for certain types of vehicles.
  \item \emph{Passenger Counts:} We'll look at the number of passengers on each trip for each service type. This
    analysis can show typical ridership patterns and how each vehicle category is used by individuals or groups.
\end{itemize}
In this section, we'll use visual aids to clearly show our findings. These visual representations will help us identify
important patterns, differences, and relationships within the data. Our goal is to find important information that can
help businesses and city officials make decisions and create plans for urban transportation in New York City.

\subsection{Temporal Trends}

\subsubsection{Hourly Analysis}

\begin{figure}[htbp]
  \label{fig:hourly-line-plot}
  \centering
  \includesvg[width=\linewidth]{figures/normalized_hourly_trip_lineplot.svg}
  \caption{Normalized number of trips per hour of day}
\end{figure}

Yellow Taxis show a steady increase in rides from the early morning, with a prominent evening peak. Green Taxis have a
similar pattern to Yellow Taxis, with a strong evening rush hour, and they have proportionaly the highest number of
rides during this time. FHVs also have their busiest time at 18:00., with a slightly earlier low point compared to taxis.
FHVHVs show a pattern that's very similar to other FHV services, with a strong evening peak and a gradual increase
throughout the day.

\begin{itemize}
  \item \emph{Peak Hours:}  All taxi and for-hire vehicle (FHV) services (like Yellow Taxi, Green Taxi, FHV, and FHVHV)
    are very busy around 18:00 This suggests that there is a strong connection between the end of the typical workday
    and evening commutes.
  \item \emph{Lowest Activity:} The lowest activity for all services is observed in the early morning hours, specifically
    between 03:00 and 05:00. This is consistent with the observation that demand for transportation is typically minimal
    during overnight hours.
  \item \emph{Morning Rush:} There are more rides across all services from 05:00 or 06:00, with a smaller peak in the
    morning around 08:00 - 09:00, probably because of the morning commute.
  \item \emph{Afternoon/Evening Surge:} After a small decrease around lunchtime, there is a big increase in rides that
    builds up through the afternoon, reaching its highest point in the evening.
\end{itemize}

\subsubsection{Daily Analysis}

All services demonstrate higher demand on Fridays and Saturdays, reflecting New York City's vibrant weekend activity.
Green Taxis and FHVHVs exhibited the most significant increase in rides on Saturday compared to other days, potentially
due to increased demand for weekend leisure and social activities, or operating more heavily in residential areas with
higher weekend demand. While Yellow Taxis also experience high demand on Saturday, their peak is on Friday, indicating
strong utilisation during the start of the weekend and potentially for business-related evening events.

\begin{figure}[htbp]
  \label{fig:daily-bar-plot}
  \centering
  \includesvg[width=\linewidth]{figures/normalized_daily_trip_bar.svg}
  \caption{Normalized number of trips per day of week}
\end{figure}

\begin{itemize}
  \item \emph{Weekend Peaks:} All services show a clear trend of increasing normalised ride counts towards the end of
    the week, with Saturday generally being the busiest day. This shows that more people need to travel during the weekend.
  \item \emph{Mid-Week Consistency:} The number of rides is fairly steady from Monday to Wednesday for most services,
    but then it goes up a lot towards the weekend.
  \item \emph{Sunday Activity:} Sunday is usually busier than the early weekdays, but less busy than Friday or Saturday,
    which suggests a different pattern of usage compared to the workweek.
\end{itemize}

\subsubsection{Monthly Analysis}

While Yellow and Green Taxis show a clear decline in summer, FHV and FHVHV services either remain robust or even
increase in demand during these months. This suggests that FHV services could potentially capture demand from tourists
or alternative travel patterns during the traditional vacation season for locals. The significant increase in FHV rides
in the latter half of the year, particularly in December, is distinctive when compared to the more consistent or earlier
peaks of Yellow and Green Taxis. This could be due to holiday shopping, seasonal events, or increased corporate travel.
FHVHV services demonstrate a more stable ridership profile throughout the year, in contrast to the more significant
seasonal fluctuations observed in other services, particularly following the initial decline in January.

\begin{figure}[htbp]
  \label{fig:monthly-bar-plot}
  \centering
  \includesvg[width=\linewidth]{figures/normalized_monthly_trip_bar.svg}
  \caption{Normalized number of trips per month}
\end{figure}

\begin{itemize}
  \item \emph{Seasonal Variations:} All TLC services show different seasonal patterns, with demand going up and down
    during the year.
  \item \emph{Summer Dip (Yellow and Green Taxis):} In the summer months (July and August), there are usually fewer
    people using Yellow and Green Taxis. This could be because New Yorkers are leaving the city for vacations or because
    there is less going on in the city.
  \item \emph{FHV Summer Peak/Stability:} On the other hand, FHV and FHVHV services have a lot of passengers in the
    summer (June, July, August and September for FHV; generally the same for FHVHV). This might suggest they are popular
    with a different type of person or are less affected by typical summer slowdowns, possibly because of tourist
    activity or people travelling for fun.
  \item \emph{Autumn and Winter Increase:} Demand generally increases in the autumn months (September, October) and
    remains strong into the early winter (November, December), with some services peaking in these months.
  \item \emph{Early Year Fluctuations:} The months of January and February demonstrated a diverse range of patterns
    across various services. Some services experienced a decline in activity, for instance, FHV in February and March.
    In contrast, other services demonstrated resilience and remained relatively strong.
\end{itemize}

\subsection{Spatial Trends}

\subsection{Pickup Locations}

\begin{figure}[htbp]
  \label{fig:most-popular-pickup-locations}
  \centering
  \includesvg[width=\linewidth]{figures/pickup_locations_map.svg}
  \caption{The map illustrates the most popular locations for picking up customers in New York}
\end{figure}

\subsubsection*{Yellow Taxis}
The top pickup locations for Yellow Taxis clearly indicate a focus on Manhattan's central business districts, major
transit hubs, and high-density tourist areas. You will find these zones around Midtown, Lower Manhattan and key
hotel/attraction vicinities. This is in line with their traditional role as street-hail vehicles serving the most
concentrated commercial and tourist parts of the city. Overlap with FHV/FHVHV (e.g., 79, 161, 230, 234) indicates shared
demand in highly active zones.

\subsubsection*{Green Taxis} The popularity of Green Taxis' locations is indicative of their mandate to serve areas
outside Manhattan's core,
primarily the outer boroughs (e.g., parts of Queens, Brooklyn, Bronx). These are likely to be local commercial centres,
dense residential neighbourhoods, and transit points within those boroughs. The difference top locations
for Yellow Taxis compared to Green taxis shows that they provide a regulated taxi service in areas where yellow
cabs are less common.

\subsubsection*{FHV and FHVHV} The areas covered by FHV and FHVHV are very similar to each other, as well as to the
popular zones of Yellow Taxi (e.g. 79, 161, 230, 234, 48). This means that these services are very competitive in
Manhattan's main area. The presence of identifiers such as 132, 138, and 61 (for FHV/FHVHV) suggests that these services
also have a very strong presence in areas that may be less saturated by traditional taxis or represent key transit
connections, such as airports. These identifiers are associated with major transportation hubs. Specifically, locations
132 and 138 correspond to two airports within the state of New York, while location 61 is situated in Highland Park. The
FHVHV list, designed for "high-volume" services such as Uber and Lyft, incorporates a range of zones that reflect high
demand across diverse urban contexts. These include areas served by Yellow Taxis and those where ride-sharing has become
predominant.

The data clearly reinforces the intended market segmentation between Yellow and Green Taxis, with Yellow dominating
Manhattan's core and Green dominating outer borough hubs.

It appears that FHV and FHVHV services are in direct competition with Yellow Taxis in areas of high demand in Manhattan.
However, they also maintain a strong presence in other areas, including those traditionally served by Green Taxis or
less adequately served by traditional taxi services.

The fact that different services have different top pickup locations (with some overlaps) indicates that they
collectively cater to the different transportation needs of New York City's residents and visitors across
different areas and times of day

\subsection{Dropoff Locations}

\begin{figure}[htbp]
  \label{fig:most-popular-dropoff-locations}
  \centering
  \includesvg[width=\linewidth]{figures/dropoff_locations_map.svg}
  \caption{The map illustrates the most popular locations for dropping off customers in New York}
\end{figure}

\subsubsection*{Yellow Taxis} The top drop-off locations for Yellow Taxis are very similar to their top pickup locations.
This suggests that Yellow Taxis are mostly used to travel within Manhattan, especially in the busiest areas. People use
Yellow Taxis to get around these areas, and mostly consist of short trips, that start and finish in the centre of Manhattan.

\subsubsection*{Green Taxi} Green Taxi drop-off locations align with their pickup patterns (similar to the Yellow Taxi
dataset), suggesting that their primary function is to facilitate travel within and between neighborhoods in the outer
boroughs. The presence of 61 (a major Queens area) as a top dropoff suggests it is a significant destination for trips
originating in other outer borough locations.

\subsubsection*{FHV and FHVHV} It is important to note that 265 (LaGuardia Airport) was the most popular place for both
FHV and FHVHV to drop off passengers, and 132 (JFK Airport) was also a popular pick. This shows that they play a big
part in getting people from the airport to their destination.

In summary, the dropoff location data highlights the key roles of Yellow and Green taxis in their respective geographic
areas, while emphasising the vital function of For-Hire Vehicles (particularly high-volume ones) in linking the whole
city to major transportation hubs like airports, as well as their pervasive presence in high-demand urban centres.

\subsection{Routes}

%=======================================================================================================================
\bibliographystyle{IEEEtran}
\begin{thebibliography}{00}

  \bibitem{zhang2019deep}
  D.~Zhang \emph{et al.}, “Deep learning + urban human mobility: A survey,”
  \emph{ACM Computing Surveys}, vol.~52, no.~5, 2019.

  \bibitem{yoro2019bigdata}
  A.~Yorozu \emph{et al.}, “Big-data analytics of taxi operations in New York
  City,” \emph{J. Advanced Transportation}, 2019.

  % — add additional references here as needed —

\end{thebibliography}

\end{document}
